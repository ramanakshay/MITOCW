%
% 6.006 problem set 0 solutions template
%
\documentclass[12pt,twoside]{article}

\input{macros-sp20}
\newcommand{\theproblemsetnum}{0}

\usepackage{amsmath}
\renewcommand{\qed}{\hfill\blacksquare}
\newcommand{\qedwhite}{\hfill \ensuremath{\Box}}

\title{6.006 Problem Set 0}

\begin{document}

\handout{Problem Set \theproblemsetnum}

\setlength{\parindent}{0pt}
\medskip\hrulefill\medskip

{\bf Name:} Akshay Raman

\medskip\hrulefill

%%%%%%%%%%%%%%%%%%%%%%%%%%%%%%%%%%%%%%%%%%%%%%%%%%%%%
% See below for common and useful latex constructs. %
%%%%%%%%%%%%%%%%%%%%%%%%%%%%%%%%%%%%%%%%%%%%%%%%%%%%%

% Some useful commands:
% $f(x) = \Theta(x)$
% $T(x, y) \leq \log(x) + 2^y + \binom{2n}{n}$
% \ttt{code\_function}


% You can create unnumbered lists as follows:
% \begin{itemize}
%     \item First item in a list
%         \begin{itemize}
%             \item First item in a list
%                 \begin{itemize}
%                     \item First item in a list
%                     \item Second item in a list
%                 \end{itemize}
%             \item Second item in a list
%         \end{itemize}
%     \item Second item in a list
% \end{itemize}

% You can create numbered lists as follows:
% \begin{enumerate}
%     \item First item in a list
%     \item Second item in a list
%     \item Third item in a list
% \end{enumerate}

% You can write aligned equations as follows:
% \begin{align}
%     \begin{split}
%         (x+y)^3 &= (x+y)^2(x+y) \\
%                 &= (x^2+2xy+y^2)(x+y) \\
%                 &= (x^3+2x^2y+xy^2) + (x^2y+2xy^2+y^3) \\
%                 &= x^3+3x^2y+3xy^2+y^3
%     \end{split}
% \end{align}

% You can create grids/matrices as follows:
% \begin{align}
%     A =
%     \begin{bmatrix}
%         A_{11} & A_{21} \\
%         A_{21} & A_{22}
%     \end{bmatrix}
% \end{align}

\begin{problems}

\problem  % Problem 1

\begin{problemparts}

%\[ A = \{ \}, B = \{ \} \]

\problempart \( A \cap B = \{ 6, 12 \}\) % Problem 1a
\problempart \( |A \cup B| = | \{ 1,3,6,9,12,13,15 \} | = 7\) % Problem 1b
\problempart \( |A - B| = |\{1,9,13\}| = 3\) % Problem 1c

\end{problemparts}

\problem  % Problem 2

\begin{problemparts}
\problempart \(E[X] = \frac{1*0 + 3*1 + 3*2 + 1*3}{8} = 1.5 \)% Problem 2a
\problempart \( E[Y] = \frac{21*21}{36} = 12.25 \) % Problem 2b
\problempart \( E[X+Y] = E[X] + E[Y] = 13.75\) % Problem 2c
\end{problemparts}

\problem  % Problem 3

\begin{problemparts}
\problempart \textbf{True. }\( 100 \equiv 18 \pmod{2} \iff 2 \vert 100-18 \iff 2 \vert 82\) % Problem 3a

\problempart \textbf{False. }\( 100 \equiv 18 \pmod{3} \iff 3 \vert 100-18 \iff 3 \vert 82\)% Problem 3b

\problempart \textbf{False. }\( 100 \equiv 18 \pmod{4} \iff 4 \vert 100-18 \iff 4 \vert 82\)% Problem 3c
\end{problemparts}

\problem  % Problem 4

\textit{Proof by induction.}

For all n  \in \mathbb{N},

\[ P(n) := \sum_{i=1}^n{i^3} = \left[ \frac{n(n+1)}{2} \right]^2 \]

\textbf{Base Case:} P(0) is true, both sides evaluate to 0.

\textbf{Inductive Step:} Assume P(n) is true, where n is a non-negative integer.

Then
\[
	\begin{split}
		\sum_{i=1}^n i^3 + (n+1)^3  &= \left[ \frac{n(n+1)}{2} \right]^2 + (n+1)^3 \\
			&= \frac{(n+1)^2(n^2 + 4n + 4)}{4} \\
			&= \left[ \frac{(n+1)(n+2)}{2} \right]^2
	\end{split}
\]

which proves P(n+1) is true.

So it follows by induction that P(n) is true for all non-negative n. \qed

\newpage

\problem  % Problem 5
	\textit{Proof by induction.}
	
	Let P(n) be the proposition that, for all \( n \geq 1\), any connected graph \(G = (V,E)\) with \(|V| = n\) and \(|E| = n-1\) is acyclic.
	
	\textbf{Base Case:} P(1) is true, since that graph only has a single vertex and no edges. 
	
	\textbf{Inductive Step:} Assume P(n) is true, where any connected graph G(n) with n vertices and n-1 edges is acyclic. Consider a connected graph G(n+1) with n+1 vertices and n edges. Since G(n+1) is connected, all vertices are connected by an edge. The average degree is \(2n/(n+1) < 2\). So there must be at least one vertex with degree 1. This vertex cannot be part of a cycle since nodes in a cycle require a degree \( \geq 2\). Removing this (weakest) vertex and the associated edge yields a graph in G(n) which is acyclic by P(n). So, P(n+1) is also true.
	
	So it follows by induction that P(n) is true for all \( n \geq 1 \). \qed
\vfill

\problem  % Problem 6
Submit your implementation to {\small\url{alg.mit.edu}}.

\begin{lstlisting}
def count_long_subarray(A):
    '''
    Input:  A     | Python Tuple of positive integers
    Output: count | number of longest increasing subarrays of A
    '''
    count = 1
    length = 1 	# max size
    current = 1 #current subarray size
    for i in range(1, len(A)):
        if A[i-1] < A[i]:
            current += 1
        else:
            current = 1
            
        if current == length:
            count += 1
        elif current > length:
            length = current
            count = 1
    return count
    
\end{lstlisting}

\end{problems}

\end{document}
