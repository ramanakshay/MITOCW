%
% 6.006 problem set 1 solutions template
%
\documentclass[12pt,twoside]{article}

\input{macros-sp20}

\usepackage{listings}

\newcommand{\theproblemsetnum}{1}

\title{6.006 Problem Set 1}

\begin{document}

\handout{Problem Set \theproblemsetnum}

\setlength{\parindent}{0pt}
\medskip\hrulefill\medskip

{\bf Name:} Akshay Raman

\medskip\hrulefill

%%%%%%%%%%%%%%%%%%%%%%%%%%%%%%%%%%%%%%%%%%%%%%%%%%%%%
% See below for common and useful latex constructs. %
%%%%%%%%%%%%%%%%%%%%%%%%%%%%%%%%%%%%%%%%%%%%%%%%%%%%%

% Some useful commands:
%$f(x) = \Theta(x)$
%$T(x, y) \leq \log(x) + 2^y + \binom{2n}{n}$
% {\tt code\_function}


% You can create unnumbered lists as follows:
%\begin{itemize}
%    \item First item in a list
%        \begin{itemize}
%            \item First item in a list
%                \begin{itemize}
%                    \item First item in a list
%                    \item Second item in a list
%                \end{itemize}
%            \item Second item in a list
%        \end{itemize}
%    \item Second item in a list
%\end{itemize}

% You can create numbered lists as follows:
%\begin{enumerate}
%    \item First item in a list
%    \item Second item in a list
%    \item Third item in a list
%\end{enumerate}

% You can write aligned equations as follows:
%\begin{align}
%    \begin{split}
%        (x+y)^3 &= (x+y)^2(x+y) \\
%                &= (x^2+2xy+y^2)(x+y) \\
%                &= (x^3+2x^2y+xy^2) + (x^2y+2xy^2+y^3) \\
%                &= x^3+3x^2y+3xy^2+y^3
%    \end{split}
%\end{align}

% You can create grids/matrices as follows:
%\begin{align}
%    A =
%    \begin{bmatrix}
%        A_{11} & A_{21} \\
%        A_{21} & A_{22}
%    \end{bmatrix}
%\end{align}

% You can include images and PDFs as follows:
% \includegraphics[width=0.5\textwidth]{img.jpg}

\begin{problems}

\problem  % Problem 1

\begin{problemparts}
\problempart \(\{f_5, f_3, f_4, f_1, f_2\}\). Using logarithm and exponential rules, we can simplify the functions. $f_1 = \Theta(n\log n)$, $f_2 = \Theta((\log n)^n)$, $f_3 = \Theta(\log n)$, $f_4 = o(n)$, $f_5 = \Theta(\log \log n)$

\problempart \(\{f_1, f_2, f_5, f_4, f_3\}\). Convert all exponents to the same base (2), $f_1 = \Theta(2^n)$, $f_2 = \Theta(2^{{(\log{6006})}^n})$, $f_3 = \Theta(2^{6006^n})$, $f_4 = \Theta(2^{{(\log {6006})} {2^n}})$, $f_5 = \Theta(2^{(\log{6006}){n^2}})$.

\problempart \(\{\{f_2, f_5\}, f_4, f_1, f_3\}\). Using sterling's approximation and formula for n choose k. $f_1 = \Theta(n^n)$, $f_2 = \Theta(n^6)$, $f_3 = \Theta(\sqrt n {\left ( \frac {6}{e} \right )}^{6n} n^{6n})$, $f_4 = \Theta({(6/5^{5/6})}^n/\sqrt n)$, $f_5 = \Theta(n^6)$ 

\problempart \(\{f_5, f_2, f_1, f_3, f_4\}\) Exponent term dominates the base. Can take logarithm of all function to understand better. $f_1 = \Theta(n^{n+4})$, $f_2 = \Theta(n^{7 \sqrt n})$, $f_3 = \Theta(n^{6n})$, $f_4 = \Theta(7^{n^2})$, $f_5 = \Theta(n^{12} n^{1/n})$

\end{problemparts}

\newpage
\problem  % Problem 2

\begin{problemparts}
\problempart % Problem 2a

% Description and Correctness

Thinking recursively, to reverse a sub-sequence with k elements, we can swap the ends i.e. elements at indices $i$ and $i+(k-1)$, a recursively solve the sub-problems. For the base case $k<2$, no work needs to be done. The correctness of the algorithm can be proved using induction.

The swap can be done by removing the elements in the reverse order (right end then left end). This will preserve the index values while deleting. Then we can insert in the correct order (left end then right end). So make use of index values from before.

% Running Time

Each swap performs four $O(\log n)$-time operations, so it happens in $O(\log n)$ time. At most, $k/2$ recursive calls are made which is $O(k)$. Therefore, the running time of the algorithm is $O(k\log n)$

\begin{lstlisting}[language=Python]
REVERSE(D, i, k):
	if k < 2:
		return None
	xr = D.delete_at(i+k-1)
	xl = D.delete_at(i)
	D.insert_at(i, xr)
	D.insert_at(i+k-1, xl)
	REVERSE(D, i+1, k-2)
\end{lstlisting}

\problempart % Problem 2b

Thinking recursively, to move a sub-sequence with k elements, we can move the first item at index i in front of index j and then recursively move the sub-sequence of size $(k-1)$ in front of that. For the base case $k = 0$, we don't have to move anything. If we maintain that: $i$ is the starting element of the subsequence, $j$ is index of the item in front of which we have to place subsequence, and $k$ is the size of the subsequence, we have prove that algorithm works correctly by induction.

After removing an item at index i. If $j>i$, the value of j decreases by 1 i.e. $j = j-1$. Also, when we insert an item after index $j$ and $j<i$, the entire subsequence shift to the right by one i.e. $i = i+1$. 

The recursive procedure make no more than $O(k)$ recursive call. In each call, it does $O(\log n)$ work. Therefore, the running time of the algorithm is $O(k \log n)$.

\begin{lstlisting}[language=Python]
MOVE(D, i, k, j):
	if k < 1:
		return None
	x = D.delete_at(i)
	if (j>i):
		j -= 1
	D.insert_at(j+1)
	j += 1
	if (j<i):
		i+=1
	MOVE(D, i, k-1, j)
\end{lstlisting}


\end{problemparts}

\newpage
\problem  % Problem 3

Store $n$ pages in a static array $S$ of size $3n$. This can be build in $O(n)$ time whenever $build(X)$ or \lstinline{place_mark(i, m)} is called. $S$ is divided into three sub-arrays $P_1, P_2, P_3$ of size $n$ with the bookmarks at the divisions. We maintain some indices: $a_s=0$, $a_e$ points to the end of $P_1$, $b_s$ points of the start of $P_2$, $b_e$ points to the end of $P_2$, $c_s$ points of the start of $P_3$, $c_e$ points to the end of $P_3$. 

To support \lstinline{read_page(i)}, we calculate the actual index of the element using the pointer from above:

\begin{itemize}
	\item \quad If $i < |P_1| $, return $S[i]$.
	\item \quad If $|P_1| \leq i < |P_1| + |P_2|$, return $S[b_s + i - a_e]$.
	\item \quad Otherwise, return $S[c_s + i - a_e - (b_e - b_s)]$.  
\end{itemize}

The running time of this operation is worst-case $O(1)$.

The procedure \lstinline{shift_mark(m,d)} involves moving the bookmark element one position left or right in the array. If we move bookmark A to the left, the rightmost element in $P_1$ becomes the first element in $P_2$. So we only have to make a single swap in $\Theta(1)$ and update pointers to preserve the invariant. Similarly, we can show that all such operations can be done in worst case $O(1)$ running time.

\lstinline{move_page(m)} will take $O(1)$ time to move elements and update the pointers. However, if any sub-array $P_1, P_2, P_3$ exceeds its limit ($n$), we will have to rebuild the array with extra space. Since this rebuilding happens once every $n$ operations, the running time of this procedure is amortized $O(1)$ time.




\newpage
\problem  % Problem 4

\begin{problemparts}
\problempart % Problem 4a

\begin{itemize}
	\item  \lstinline{insert_first(x)}: Create a new node x. If the list is empty, set both $L.head$ and $L.tail$ to x. Otherwise, connect node $x$ to $L.head$ by updating their next and prev pointers. Then, we set $L.head$ to the newly created node x.
	\item \lstinline{insert_last(x)}: Same as \lstinline{insert_first(x)} but this time we update $L.tail$ instead.
	\item \lstinline{delete_first()}: Extract and store the head node from the list. Then update $L.head$ to the next node in the list. If there is no next node, then set $L.head$ to None.
	\item \lstinline{delete_last()}: Extract and store the tail node from the list. Then update $L.tail$ to the previous node in the list. If there is no previous node, then set $L.tail$ to None.
\end{itemize}


\problempart % Problem 4b
Construct a new list by setting $x_1$ and $x_2$ as the head and tail respectively. When $x_1$ or $x_2$ are ends of $L$, care must be taken to update $L.head$ and $L.tail$. If $L.head = x_1$, then set the new head to be the next node of $x_2$ (called $a$). Otherwise, set the next node of $x_1$'s previous node to $a$. If $L.tail = x_2$, then set the new tail to be the previous node of $x_1$ (called $b$). Otherwise, set the previous node of $x_2$'s next node to $b$. This algorithm remove nodes between $x_1$ and $x_2$ directly so it is correct. The running time is $O(1)$ since be make a constant number of pointer updates. 

\problempart % Problem 4c
 To splice into $L_1$, first store the next value of $x$ in variable $x_{next}$. Then, set the next node of $x$ to $L_2.head$ (and previous node of $L_2.head$ to $x$). Also, we set previous node of $x_{next}$ to $L_2.tail$ (and next node of $L_2.tail$ to $x$). To make $L_2$ empty, we can simply set its head and tail to point to None. If $x_{next}$ is None, we set $L_1.tail$ to $L_2.tail$. This algorithm splices $L_2$ into $L_1$ directly, so it is correct. Since we make a constant number of pointer updates, the running
 
\problempart This python implementation for all operations is given below:

\lstinputlisting[language=Python]{Doubly_Linked_List_Seq.py}
\end{problemparts}

\end{problems}

\end{document}
